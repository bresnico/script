% Options for packages loaded elsewhere
\PassOptionsToPackage{unicode}{hyperref}
\PassOptionsToPackage{hyphens}{url}
%
\documentclass[
  french,
]{article}
\usepackage{amsmath,amssymb}
\usepackage{lmodern}
\usepackage{ifxetex,ifluatex}
\ifnum 0\ifxetex 1\fi\ifluatex 1\fi=0 % if pdftex
  \usepackage[T1]{fontenc}
  \usepackage[utf8]{inputenc}
  \usepackage{textcomp} % provide euro and other symbols
\else % if luatex or xetex
  \usepackage{unicode-math}
  \defaultfontfeatures{Scale=MatchLowercase}
  \defaultfontfeatures[\rmfamily]{Ligatures=TeX,Scale=1}
\fi
% Use upquote if available, for straight quotes in verbatim environments
\IfFileExists{upquote.sty}{\usepackage{upquote}}{}
\IfFileExists{microtype.sty}{% use microtype if available
  \usepackage[]{microtype}
  \UseMicrotypeSet[protrusion]{basicmath} % disable protrusion for tt fonts
}{}
\makeatletter
\@ifundefined{KOMAClassName}{% if non-KOMA class
  \IfFileExists{parskip.sty}{%
    \usepackage{parskip}
  }{% else
    \setlength{\parindent}{0pt}
    \setlength{\parskip}{6pt plus 2pt minus 1pt}}
}{% if KOMA class
  \KOMAoptions{parskip=half}}
\makeatother
\usepackage{xcolor}
\IfFileExists{xurl.sty}{\usepackage{xurl}}{} % add URL line breaks if available
\IfFileExists{bookmark.sty}{\usepackage{bookmark}}{\usepackage{hyperref}}
\hypersetup{
  pdftitle={Script du cadre théorique avec références},
  pdfauthor={Nicolas Bressoud},
  pdflang={fr},
  hidelinks,
  pdfcreator={LaTeX via pandoc}}
\urlstyle{same} % disable monospaced font for URLs
\usepackage[margin=1in]{geometry}
\usepackage{longtable,booktabs,array}
\usepackage{calc} % for calculating minipage widths
% Correct order of tables after \paragraph or \subparagraph
\usepackage{etoolbox}
\makeatletter
\patchcmd\longtable{\par}{\if@noskipsec\mbox{}\fi\par}{}{}
\makeatother
% Allow footnotes in longtable head/foot
\IfFileExists{footnotehyper.sty}{\usepackage{footnotehyper}}{\usepackage{footnote}}
\makesavenoteenv{longtable}
\usepackage{graphicx}
\makeatletter
\def\maxwidth{\ifdim\Gin@nat@width>\linewidth\linewidth\else\Gin@nat@width\fi}
\def\maxheight{\ifdim\Gin@nat@height>\textheight\textheight\else\Gin@nat@height\fi}
\makeatother
% Scale images if necessary, so that they will not overflow the page
% margins by default, and it is still possible to overwrite the defaults
% using explicit options in \includegraphics[width, height, ...]{}
\setkeys{Gin}{width=\maxwidth,height=\maxheight,keepaspectratio}
% Set default figure placement to htbp
\makeatletter
\def\fps@figure{htbp}
\makeatother
\setlength{\emergencystretch}{3em} % prevent overfull lines
\providecommand{\tightlist}{%
  \setlength{\itemsep}{0pt}\setlength{\parskip}{0pt}}
\setcounter{secnumdepth}{5}
\ifxetex
  % Load polyglossia as late as possible: uses bidi with RTL langages (e.g. Hebrew, Arabic)
  \usepackage{polyglossia}
  \setmainlanguage[]{french}
\else
  \usepackage[main=french]{babel}
% get rid of language-specific shorthands (see #6817):
\let\LanguageShortHands\languageshorthands
\def\languageshorthands#1{}
\fi
\ifluatex
  \usepackage{selnolig}  % disable illegal ligatures
\fi
\newlength{\cslhangindent}
\setlength{\cslhangindent}{1.5em}
\newlength{\csllabelwidth}
\setlength{\csllabelwidth}{3em}
\newenvironment{CSLReferences}[2] % #1 hanging-ident, #2 entry spacing
 {% don't indent paragraphs
  \setlength{\parindent}{0pt}
  % turn on hanging indent if param 1 is 1
  \ifodd #1 \everypar{\setlength{\hangindent}{\cslhangindent}}\ignorespaces\fi
  % set entry spacing
  \ifnum #2 > 0
  \setlength{\parskip}{#2\baselineskip}
  \fi
 }%
 {}
\usepackage{calc}
\newcommand{\CSLBlock}[1]{#1\hfill\break}
\newcommand{\CSLLeftMargin}[1]{\parbox[t]{\csllabelwidth}{#1}}
\newcommand{\CSLRightInline}[1]{\parbox[t]{\linewidth - \csllabelwidth}{#1}\break}
\newcommand{\CSLIndent}[1]{\hspace{\cslhangindent}#1}

\title{Script du cadre théorique avec références}
\author{Nicolas Bressoud}
\date{janvier 2021}

\begin{document}
\maketitle

\renewcommand*\contentsname{Table des matières}
{
\setcounter{tocdepth}{2}
\tableofcontents
}
\hypertarget{inclusion}{%
\section{Inclusion}\label{inclusion}}

\hypertarget{contexte-et-cadrage-large}{%
\subsection{Contexte et cadrage large}\label{contexte-et-cadrage-large}}

(HOT, revue théorique) Le terme d'inclusion s'impose dans le langage public à la place du terme intégration. L'inclusion scolaire implique une vision postnormative de l'éducation (Ebersold, 2009).

\hypertarget{inclure-cest-diffuxe9rencier}{%
\subsection{Inclure c'est différencier}\label{inclure-cest-diffuxe9rencier}}

(livre) Si l'inclusion appelle à une prise en compte des différences, il s'agit de prendre en compte la capacité du système scolaire à s'adapter. Cette question occupe les chercheurs ou penseurs en éducation depuis de nombreuses années à partir d'un autre cadre de référence. Philippe Perrenoud (2012) fait partie des personnes francophones qui ont fortement influencé et augmenté notre compréhension du fonctionnement scolaire et des ses caractéristiques normalisantes.

\hypertarget{pratiques}{%
\subsection{Pratiques}\label{pratiques}}

(Livre) Les pratiques liées à l'inclusion scolaire sont variées et peu équivalentes en termes de qualité. Des auteurs comme par exemple Tremblay (2012) proposent des revues intéressantes des pratiques courantes.

\hypertarget{bonheur-bien-uxeatre-etc}{%
\section{Bonheur, bien-être, etc}\label{bonheur-bien-uxeatre-etc}}

\hypertarget{bonheur-uxe9pistuxe9mologie-querelles}{%
\subsection{Bonheur, épistémologie, querelles}\label{bonheur-uxe9pistuxe9mologie-querelles}}

(Revue théorique) Les querelles philosophiques autour de la notion de bonheur restent vives (voir, par exemple, Ferry, 2016).

(livre) Bienveillance, bonheur ou bien-être semblent devenir des termes courant dans les discussions autour de l'école. Mais de quoi parle-t-on ? Si la bienveillance a une place certaine, elle doit être vue comme une attitude de reconnaissance et d'exigence. Jellab \& Marsollier (2018) proposent un cadre et donne la parole à des personnes du terrain.

\hypertarget{bien-uxeatre-scolaire-duxe9finition-et-cadre}{%
\subsection{Bien-être scolaire : définition et cadre}\label{bien-uxeatre-scolaire-duxe9finition-et-cadre}}

Au niveau mondial, des chercheuses telles que Suldo (2016) ont énormément contribué à démocratiser la notion de bien-être scolaire en proposant des programmes à large échelle.

\hypertarget{forces-de-caractuxe8re}{%
\section{Forces de caractère}\label{forces-de-caractuxe8re}}

\emph{Fouiller le site VIA.}

\hypertarget{en-guxe9nuxe9ral}{%
\subsection{En général}\label{en-guxe9nuxe9ral}}

(Livre) Travailler sur les aspects positifs de l'individu n'est ni nouveau ni original. Parler de talent, d'aisance ou de «~son~» élément est au coeur des démarches de développement personnel (voir par exemple Robinson et al., 2015).

(Livre) La perspective systémique illustre bien en quoi le repérage et l'activation de ressources, au sein d'un réseau, peut faciliter une évolution favorable des relations (Curonici et al., 2014).

\hypertarget{en-contexte-scolaire}{%
\subsection{En contexte scolaire}\label{en-contexte-scolaire}}

(Étude) En Austtralie, un coaching sur les forces développe l'engagement et l'espoir chez des jeunes de environ 10-11 ans (Madden et al., 2011).

(HOT, revue de pratiques) The Good School. En Australie, chez les jeunes 12 ans et moins, différentes intégrations des forces sont présentées. Les chercheurs ne se proposent pas de mesurer les effets (White et al., 2015a).

(Revue de pratiques) L'intégration des forces dans la vie scolaire est une pratique pertinente dans les écoles du XXIème siècle (Lavy, 2020).

\hypertarget{perspectives-guxe9nuxe9rales}{%
\subsection{Perspectives générales}\label{perspectives-guxe9nuxe9rales}}

(Proposition théorique) Identifier et utiliser les forces ne suffit pas. La dynamique contextuelle doit être prise en compte (Biswas-Diener et al., 2011).

\hypertarget{pratiques-1}{%
\subsection{Pratiques}\label{pratiques-1}}

(Livre) Les pratiques concernant les forces sont nombreuses et variées (voir par exemple Niemiec, 2017).

\hypertarget{uxe9ducation-positive}{%
\section{Éducation positive}\label{uxe9ducation-positive}}

En contexte anglophone, l'éducation positive constitue un domaine pédagogique très sérieux, objet de recherches intenses. Les pratiques pédagogiques proposées sont documentées par des publications scientifiques (White et al., 2015b).

\hypertarget{contextualisation-internationale-ou-bonnes-pratiques}{%
\subsection{Contextualisation internationale ou bonnes pratiques}\label{contextualisation-internationale-ou-bonnes-pratiques}}

(livres) De nombreuses ressources existent pour proposer des interventions éducatives basées sur ce qu'une science du bien-être propose. Par exemple, Harlé (2017) ou Boniwell \& Reynaud (2018) consacrent des ouvrages francophones sur la question en mêlant conseils, pistes et éclairages théoriques.

(Livre) Le terme éducation positive étant connoté, des auteurs n'hésitent pas à utliser d'autres termes (par exemple, Dini \& Scanziani, 2016).

\hypertarget{besoins-uxe9ducatifs-particuliers}{%
\section{Besoins éducatifs particuliers}\label{besoins-uxe9ducatifs-particuliers}}

(Livre) La démarche de projet pédagogique individuel reste une manière courante de travailler sur les singularités des situtations. Cette démarche est à voir essentiellement en tant que procédure de résolution de problèmes (Vianin, 2016).

\hypertarget{contexte-du-handicap-vocabulaire-concepts}{%
\subsection{Contexte du handicap, vocabulaire, concepts}\label{contexte-du-handicap-vocabulaire-concepts}}

(Proposition théorique) Dans cette approche biopsychosociale, la prise en compte du continuum processuel modifie nos conceptions de la notion de handicap mental et ne la résume pas à une limitation biologique (Kinderman, 2005).

\hypertarget{environnement-actuel}{%
\subsection{Environnement actuel}\label{environnement-actuel}}

\hypertarget{psychologie-positive}{%
\section{Psychologie positive}\label{psychologie-positive}}

\hypertarget{duxe9finition-et-fondements}{%
\subsection{Définition et fondements}\label{duxe9finition-et-fondements}}

(livre) La psychologie positive est en courant de recherche orienté vers le développement du fonctionnement. Il peut s'inscrire dans une culture humaniste (Shankland \& Lantheaume, 2018 ; Lecomte, 2012 ; Martin-Krumm \& Tarquinio, 2011).

(Livre) La psychologie positive ne saurait se réduire à une «~science du positif~». Ce courant s'intéresse en profondeur au fonctionnement humain dans sa globalité. Par exemple, si l'optimisme a longtemps été considéré comme incontournable, l'étude du pessimisme et ses potentiels bienfaits est aussi présente (Norem \& Boniwell, 2015).

\hypertarget{controverses}{%
\subsection{Controverses}\label{controverses}}

\hypertarget{uxe9tudes-de-base}{%
\subsection{Études de base}\label{uxe9tudes-de-base}}

(Étude) Le rôles des émotions positives est mis en évidence dans les changements cognitifs des programmes de psychologie positive (Gander et al., 2018).

\hypertarget{a-luxe9cole}{%
\subsection{A l'école}\label{a-luxe9cole}}

(Étude) Chez les adolescents, l'intégration des stratégies de régulation émotionnelles joue un rôle prépondérant dans le développement du bien-être (Burckhardt et al., 2016).

\hypertarget{climat-de-classe}{%
\section{Climat de classe}\label{climat-de-classe}}

\hypertarget{climat-duxe9tablissement-et-santuxe9-psychique}{%
\subsection{Climat d'établissement et santé psychique}\label{climat-duxe9tablissement-et-santuxe9-psychique}}

\hypertarget{relations-et-enjeux-uxe9motionnels-avec-lenseignante}{%
\subsection{Relations et enjeux émotionnels avec l'enseignant·e}\label{relations-et-enjeux-uxe9motionnels-avec-lenseignante}}

(Proposition théorique) La manière dont les enseignant·es communiquent les émotions, particulièrement dans les petits degrés, est une question importante de la recherche (Yelinek \& Grady, 2017).

(Étude) Chez les adolescent·es, une qualité de relation à l'enseignant·e prédit moins de problèmes de comportement et des attitudes prosociales (Obsuth et al., 2017).

(livre) La qualité de la relation à l'enseignant·e est bien reliée au bien-être et aux performances académiques. Sur ce dernier point, on peut consulter les travaux de synthèse de Hattie \& Clarke (2019).

\hypertarget{gestion-de-classe}{%
\subsection{Gestion de classe}\label{gestion-de-classe}}

(Méta-analyse) La gestion de classe influence les résultats des élèves, notamment sur les plans motivationnels et émotionnels (Korpershoek et al., 2016).

(Livre) La gestion de classe peut se décomposer en 5 dimensions dont 4 constituent des mesures de prévention de l'indiscipline (Gaudreau, 2017).

\hypertarget{contexte-scolaire}{%
\section{Contexte scolaire}\label{contexte-scolaire}}

\hypertarget{contexte-luxe9gal}{%
\subsection{Contexte légal}\label{contexte-luxe9gal}}

\hypertarget{contexte-romand-ou-valaisan}{%
\subsection{Contexte romand ou valaisan}\label{contexte-romand-ou-valaisan}}

\hypertarget{environnement-ou-voisinage-puxe9dagogique}{%
\subsection{Environnement ou voisinage pédagogique}\label{environnement-ou-voisinage-puxe9dagogique}}

(livre) Les débats et transformations scolaires rythment la vie de nos élèves. En reconnaissant l'impact des personnes ayant fondé des courants forts, les chercheurs actuellement contextualisent toujours leur travaux. Par exemple, Houdé (2018) a réalisé un exercice intéressant en liant neurosciences et grands courants pédagogiques classiques.

(Livre) La pleine conscience entre aussi dans les écoles avec des intentions louables, augmenter le bien-être (voir, par exemple, Kotsou, 2018).

(Livre) Une meilleure connaissance de nos biais cognitifs et mécanismes attentionnels ouvre des perspectives enthousiasmantes (Kahneman, 2015).

(Livre) La méthode scientifique et les apports, en particulier, des sciences cognitives, permettent une prise de recul nouvelle sur les habitudes pédagogiques (Brown et al., 2016).

(Livre) Des tentatives plus ou moins heureuses de connecter les pratiques, théories, croyances pédagogiques avec des sciences plus dures existent avec plus ou moins de réussite (voir par exemple Sousa et al., 2013).

\hypertarget{formation-des-enseignantes}{%
\subsection{Formation des enseignant·es}\label{formation-des-enseignantes}}

(Proposition théorique) Le travail à partir de la théorie ou les représentations ne convient pas. L'entrée resterait l'expérience (Willingham, 2017).

\hypertarget{refs}{}
\begin{CSLReferences}{1}{0}
\leavevmode\hypertarget{ref-biswas-diener2011}{}%
Biswas-Diener, R., Kashdan, T. B., \& Minhas, G. (2011). A Dynamic Approach to Psychological Strength Development and Intervention. \emph{The Journal of Positive Psychology}, \emph{6}(2), 106‑118. \url{https://doi.org/10.1080/17439760.2010.545429}

\leavevmode\hypertarget{ref-boniwell2018}{}%
Boniwell, I., \& Reynaud, L. (2018). \emph{{Parcours d'éducation positive et scientifique: les 10 étapes clés pour une éducation heureuse et épanouie}}. {Leduc.s pratique}.

\leavevmode\hypertarget{ref-brown2016}{}%
Brown, P. C., Roediger, H. L., McDaniel, M. A., Pasquinelli, E., Viguier, A., \& Randon-Furling, J. (2016). \emph{{Mets-toi ça dans la tête!: les stratégies d'apprentissage à la lumière des sciences cognitives}}. {Éditions Markus Haller}.

\leavevmode\hypertarget{ref-burckhardt2016}{}%
Burckhardt, R., Manicavasagar, V., Batterham, P. J., \& Hadzi-Pavlovic, D. (2016). A Randomized Controlled Trial of Strong Minds: {A} School-Based Mental Health Program Combining Acceptance and Commitment Therapy and Positive Psychology. \emph{Journal of School Psychology}, \emph{57}, 41‑52. \url{https://doi.org/10.1016/j.jsp.2016.05.008}

\leavevmode\hypertarget{ref-curonici2014}{}%
Curonici, C., Joliat, F., \& MacCulloch, P. (2014). \emph{{Des difficultés scolaires aux ressources de l'école: un modèle de consultation systémique pour psychologues et enseignants}}. {De Boeck}.

\leavevmode\hypertarget{ref-dini2016}{}%
Dini, F., \& Scanziani, E. (2016). \emph{{Une éducation intégrale pour grandir en s'épanouissant: accompagner les enfants et les adolescents avec bienveillance et discernement}}. {Faim de Siècle}.

\leavevmode\hypertarget{ref-ebersold2009}{}%
Ebersold, S. (2009).{} {Inclusion}{}. \emph{Recherche et Formation}, \emph{61}, 71‑83.

\leavevmode\hypertarget{ref-ferry2016a}{}%
Ferry, L. (2016). \emph{{7 façons d'être heureux ou Les paradoxes du bonheur}}. {XO éditions}.

\leavevmode\hypertarget{ref-gander2018}{}%
Gander, F., Proyer, R. T., \& Ruch, W. (2018). A {Placebo}-{Controlled Online Study} on {Potential Mediators} of a {Pleasure}-{Based Positive Psychology Intervention}: {The Role} of {Emotional} and {Cognitive Components}. \emph{Journal of Happiness Studies}, \emph{19}(7), 2035‑2048. \url{https://doi.org/10.1007/s10902-017-9909-3}

\leavevmode\hypertarget{ref-gaudreau2017}{}%
Gaudreau, N. (2017). \emph{Gérer Efficacement Sa Classe: Les Ingrédients Essentiels}. {Presses de l'Université du Québec}.

\leavevmode\hypertarget{ref-harle2017}{}%
Harlé, M. (2017). \emph{{Les 5 clés d'une éducation réussie !: dépassez vos préjugés!}} {Hachette Pratique}.

\leavevmode\hypertarget{ref-hattie2019}{}%
Hattie, J., \& Clarke, S. (2019). \emph{Visible Learning: Feedback}. {Routledge}.

\leavevmode\hypertarget{ref-houde2018a}{}%
Houdé, O. (2018). \emph{{L'école du cerveau: de Montessori, Freinet et Piaget aux sciences cognitives}}.

\leavevmode\hypertarget{ref-jellab2018}{}%
Jellab, A., \& Marsollier, C. (2018). \emph{{Bienveillance et bien-être à l'école: plaidoyer pour une éducation humaine et exigeante}}. {Berger-Levrault}.

\leavevmode\hypertarget{ref-kahneman2015}{}%
Kahneman, D. (2015). \emph{{Système 1, système 2: les deux vitesses de la pensée}}. {Flammarion}.

\leavevmode\hypertarget{ref-kinderman2005}{}%
Kinderman, P. (2005). A Psychological Model of Mental Disorder. \emph{Harvard Review of Psychiatry}, \emph{13}(4), 206‑217. \url{https://doi.org/10.1080/10673220500243349}

\leavevmode\hypertarget{ref-korpershoek2016}{}%
Korpershoek, H., Harms, T., de Boer, H., van Kuijk, M., \& Doolaard, S. (2016). A {Meta}-{Analysis} of the {Effects} of {Classroom Management Strategies} and {Classroom Management Programs} on {Students Academic}, {Behavioral}, {Emotional}, and {Motivational Outcomes}. \emph{Review of Educational Research}, \emph{86}(3), 643‑680. \url{https://doi.org/10.3102/0034654315626799}

\leavevmode\hypertarget{ref-kotsou2018}{}%
Kotsou, I. (2018). \emph{{La pleine conscience à l'école: de 5 ans à 12 ans.}} {De Boeck}.

\leavevmode\hypertarget{ref-lavy2020}{}%
Lavy, S. (2020). A {Review} of {Character Strengths Interventions} in {Twenty}-{First}-{Century Schools}: Their {Importance} and {How} They Can Be {Fostered}. \emph{Applied Research in Quality of Life}, \emph{15}(2), 573‑596. \url{https://doi.org/10.1007/s11482-018-9700-6}

\leavevmode\hypertarget{ref-lecomte2012c}{}%
Lecomte, J. (2012). \emph{La Bonté Humaine: {Altruisme}, Empathie, Générosité}. {Odile Jacob}.

\leavevmode\hypertarget{ref-madden2011}{}%
Madden, W., Green, S., \& Grant, A. M. (2011). A Pilot Study Evaluating Strengths-Based Coaching for Primary School Students: {Enhancing} Engagement and Hope. \emph{International Coaching Psychology Review}, \emph{6}(1), 71‑83.

\leavevmode\hypertarget{ref-martin-krumm2011}{}%
Martin-Krumm, C., \& Tarquinio, C. (2011). \emph{{Traité de psychologie positive}}. {De Boeck}.

\leavevmode\hypertarget{ref-niemiec2017b}{}%
Niemiec, R. M. (2017). \emph{Character Strengths Interventions: A Field Guide for Practitioners}. {Hogrefe}.

\leavevmode\hypertarget{ref-norem2015}{}%
Norem, J., \& Boniwell, I. (2015). \emph{{Découvrez le pouvoir positif du pessimisme!}} {InterEditions}.

\leavevmode\hypertarget{ref-obsuth2017}{}%
Obsuth, I., Murray, A. L., Malti, T., Sulger, P., Ribeaud, D., \& Eisner, M. (2017). A {Non}-Bipartite {Propensity Score Analysis} of the {Effects} of {Teacher}-{Student Relationships} on {Adolescent Problem} and {Prosocial Behavior}. \emph{Journal of Youth and Adolescence}, \emph{46}(8), 1661‑1687. \url{https://doi.org/10.1007/s10964-016-0534-y}

\leavevmode\hypertarget{ref-philippeperrenoud2012}{}%
Philippe Perrenoud. (2012). \emph{L'organisation Du Travail, Clé de Toute Pédagogie Différenciée}. {ESF}.

\leavevmode\hypertarget{ref-robinson2015}{}%
Robinson, K., Aronica, L., Robinson, K., \& Bouvier, M. (2015). \emph{{Trouver son élément: comment découvrir ses talents et ses passions pour tranformer sa vie!}}

\leavevmode\hypertarget{ref-shankland2018b}{}%
Shankland, R., \& Lantheaume, S. (2018). \emph{{La psychologie positive}}.

\leavevmode\hypertarget{ref-sousa2013}{}%
Sousa, D. A., Tomlinson, C. A., \& Sirois, G. (2013). \emph{{Comprendre le cerveau pour mieux différencier: adapter l'enseignement aux besoins des apprenants grâce aux apports des neurosciences}}. {Chenelière éducation}.

\leavevmode\hypertarget{ref-suldo2016a}{}%
Suldo, S. M. (2016). \emph{Promoting Student Happiness: Positive Psychology Interventions in Schools}. {The Guilford Press}.

\leavevmode\hypertarget{ref-tremblay2012}{}%
Tremblay, P. (2012). \emph{{Inclusion scolaire: dispositifs et pratiques pédagogiques}}. {De Boeck}.

\leavevmode\hypertarget{ref-vianin2016}{}%
Vianin, P. (2016). \emph{{Comment développer un processus d'aide pour les élèves en difficulté?: enseignants, psychologues, éducateurs, formateurs}}. {De Boeck supérieur}.

\leavevmode\hypertarget{ref-white2015a}{}%
White, M. A., Murray, A. S., \& Seligman, M. E. P. (2015a). \emph{Evidence-Based Approaches in Positive Education: Implementing a Strategic Framework for Well-Being in Schools}.

\leavevmode\hypertarget{ref-white2015c}{}%
White, M. A., Murray, A. S., \& Seligman, M. E. P. (2015b). \emph{Evidence-Based Approaches in Positive Education: Implementing a Strategic Framework for Well-Being in Schools}.

\leavevmode\hypertarget{ref-willingham2017}{}%
Willingham, D. T. (2017). A {Mental Model} of the {Learner} : {Teaching} the {Basic Science} of {Educational Psychology} to {Future Teachers}. \emph{Mind, Brain, and Education}, \emph{11}(4), 166‑175. \url{https://doi.org/10.1111/mbe.12155}

\leavevmode\hypertarget{ref-yelinek2017}{}%
Yelinek, J., \& Grady, J. S. (2017). {``{Show} Me Your Mad Faces!''} Preschool Teachers' Emotion Talk in the Classroom. \emph{Early Child Development and Care}, \emph{189}(7), 1063‑1071. \url{https://doi.org/10.1080/03004430.2017.1363740}

\end{CSLReferences}

\end{document}
